\documentclass{article}
\usepackage[utf8]{inputenc}
\usepackage[UKenglish]{babel}
\renewcommand{\familydefault}{\sfdefault}
\usepackage{ifpdf}
\usepackage{hyperref}
\title{Ceph Mon: a technical overview}
\author{Paolo VIOTTI}
\date{\today}
\ifpdf
\hypersetup{
    pdfauthor={Paolo VIOTTI},
    pdftitle={Ceph Mon: a technical overview},
}
\fi
\begin{document}

\maketitle

\begin{abstract}
Abstract
\end{abstract}

\section{Ceph: an introduction}
what is Ceph
When it started

\section{Ceph Mon}

\subsection{Architecture}


\subsection{Code}


\subsection{Additional notes}


\begin{thebibliography}{1}

  \bibitem{notes} John W. Dower {\em Readings compiled for History
  21.479.}  1991.

  \bibitem{impj}  The Japan Reader {\em Imperial Japan 1800-1945} 1973:
  Random House, N.Y.

  \bibitem{norman} E. H. Norman {\em Japan's emergence as a modern
  state} 1940: International Secretariat, Institute of Pacific
  Relations.

  \bibitem{fo} Bob Tadashi Wakabayashi {\em Anti-Foreignism and Western
  Learning in Early-Modern Japan} 1986: Harvard University Press.

\end{thebibliography}
	
\end{document}
